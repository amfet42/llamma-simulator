\documentclass[notitlepage]{revtex4-2}
\usepackage[utf8]{inputenc}
\usepackage{geometry}
\usepackage{xcolor}
\geometry{a4paper, margin=1in}
\usepackage{graphicx}
\usepackage{amsmath}
\usepackage[margin=5pt]{subfig}
\definecolor{darkgreen}{rgb}{0.00,0.50,0.25}
\definecolor{darkblue}{rgb}{0.00,0.00,0.67}
\usepackage[breaklinks,pdftitle={Optimal Parameters for LlamaLend Markets}, pdfauthor={curve.finance},colorlinks,urlcolor=blue,citecolor=darkgreen,linkcolor=darkblue]{hyperref}
\usepackage{mathtools}
\usepackage{tikz}
\usepackage{listings}
\graphicspath{{pdf/}}

\setkeys{Gin}{scale=0.5}

\begin{document}

\begin{abstract}
Proper parameter tuning is the foundation of a safe and efficient lending market. This document summarizes use the
LLAMMA (Lending--Liquidating AMM Algorithm) simulator for analyzing historical data, modeling market behavior, and
optimizing key parameters (fee, amplification parameter~$A$).
\end{abstract}

\section{Introduction}
LlamaLend markets are \emph{permissionless} --- anyone can deploy them without special approval. However, creating
a market requires expertise and responsibility; therefore, this functionality is not included in the default Curve UI
and instead uses a factory contract.
Before deployment, one should study the behavior of the collateral asset under different market regimes
(bullish/bearish trends, volatility spikes) and run simulations. The \textbf{LLAMMA simulator} helps to:
\begin{itemize}
  \item analyze historical price data;
  \item model different market scenarios under varying parameters;
  \item assess risk and profitability;
  \item optimize the key parameters ($A$, fee, and band count).
\end{itemize}

\subsection*{Quick Start}
It's recommended to use pypy to do faster simulations. Create environment and install dependencies:

\begin{verbatim}
uv python install pypy@3.11
uv venv --python pypy@3.11
uv sync -p pypy@3.11
\end{verbatim}

Add pair to simulator/settings.py and import historical data:
\begin{verbatim}
python manage.py import_data {pair_name} (i.e. BTCUSDT)
\end{verbatim}

\section{Configuration Before Simulation}
Two main simulations are used for market tuning: \texttt{2\_simulate\_A.py} (search for $A$) and
\texttt{1\_simulate\_dynamic\_fee.py} (search for fee). Despite the numbering, you should \textbf{first} find $A$,
and \textbf{then} find \texttt{fee}.

Before running, variables are adjusted for the specific asset (ready-made examples exist in the repository).

\subsection*{Key Variables}
\begin{itemize}
  \item \texttt{A=[int(a) for a in logspace(log10(10), log10(300), 50)]} Range of $A$ values scanned in \texttt{2\_simulate\_A.py}: 10 to 300 (50 points). Volatile assets require smaller $A$; correlated assets (e.g., stablecoins) use larger $A$.
  \item \texttt{A=30} Fixed $A$ value for \texttt{1\_simulate\_fee.py}.
  \item \texttt{range\_size=4} Number of bands. A minimum of 4 is recommended — representing the most conservative stress case.
  \item \texttt{dynamic\_fee=0.002} Fee for the initial $A$ scan, calculated by difference of \((p_{oracle}-p)*d\_fee\).
  \item \texttt{fee=logspace(log10(0.0001), log10(0.02), 20)} Range of fees for scanning during \texttt{1\_simulate\_fee.py}.
  \item \texttt{min\_loan\_duration, max\_loan\_duration} Time in days. Example: $1/24/2=1/48$ days $\approx$ 30 minutes. Longer duration = slower arbitrage.
  \item \texttt{samples=500000, n\_top\_samples=5} Number of Monte Carlo samples and averaging window (tail risk). For fee scans, you can lower \texttt{samples} and increase \texttt{n\_top\_samples} (e.g. $10^4$ and $10^3$).
  \item \texttt{add\_reverse=true} Adds mirrored price data (balances up/down trends).
  \item \texttt{Texp=600} EMA horizon (seconds), must match the oracle parameter (default NG pools = 600).
\end{itemize}

\section{Simulation and Parameter Optimization}
\subsection{Optimal Amplification Parameter ($A$)}
Run:
\begin{lstlisting}[language=bash]
python3 simulator/pairs/btcusd/simulate_a.py
\end{lstlisting}
The script plots two charts: losses and discount (for a given band count). The optimal $A$ is at the minimum of the
orange curve (typically around $A\approx30$).

\subsection{Optimal Fee}
Run:
\begin{lstlisting}[language=bash]
python3 simulator/pairs/btcusd/simulate_dynamic_fee.py
\end{lstlisting}
The optimal fee corresponds to the minimum of the loss curve at fixed $A$ and band count.

\subsection{Liquidation and Loan Discounts}
The \texttt{liquidation\_discount} is taken as the y-value of the blue line under the orange curve’s minimum
(e.g. $\approx0.07$). Then:
\begin{flushleft}
\[
\begin{aligned}
& \texttt{liquidation\_discount} \approx 0.07\\
%
& \texttt{loan\_discount} \approx \texttt{liquidation\_discount} + 0.03 \approx 0.10
\end{aligned}
\]
\end{flushleft}


\end{document}
